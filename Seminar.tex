\documentclass[10pt,a4paper]{article}
\usepackage[utf8]{inputenc}
\usepackage[german]{babel}
\usepackage[T1]{fontenc}
\usepackage{amsmath}
\usepackage{amsfonts}
\usepackage{amssymb}
\usepackage{hyperref}
\usepackage{graphicx}
\usepackage{xcolor}

\usepackage{amsthm}

\theoremstyle{plain}
\newtheorem{thm}{Theorem}[section]
\newtheorem{lem}[thm]{Lemma}
\newtheorem{prop}[thm]{Proposition}
\newtheorem{satz}[thm]{Satz}

\theoremstyle{definition}
\newtheorem{defn}{Definition}[section]
\newtheorem*{note}{Bemerkung}
\newtheorem*{folg}{Folgerung}
\newtheorem*{mot}{Motivation}


\theoremstyle{remark}
\newtheorem{hilfslem}{Hilfslemma}

\DeclareMathOperator{\R}{\mathbb{R}}
\DeclareMathOperator{\C}{\mathbb{C}}
\DeclareMathOperator{\Q}{\mathbb{Q}}
\DeclareMathOperator{\Z}{\mathbb{Z}}
\DeclareMathOperator{\N}{\mathbb{N}}
\DeclareMathOperator{\re}{Re}
\DeclareMathOperator{\im}{Im}
\DeclareMathOperator{\Log}{Log}
\DeclareMathOperator{\disc}{disc}
\DeclareMathOperator{\ggT}{ggT}



\newcommand\invisiblesection[1]{
  \refstepcounter{section}
  \addcontentsline{toc}{section}{\protect\numberline{\thesection}#1}
  \sectionmark{#1}}

\title{\textbf{Vortrag 14} \textbf{-- Die Dedekindsche Zetafunktion}}
\author{ \textit{\textbf{Primzahlen der Form x²+ny²}} \\ \\Seminar im WS 2021/2022}
\date{Vinzenz Baumann}
\begin{document}
\maketitle

\begin{abstract}

Wir führen die Dedekindsche Zetafunktion $\zeta_{\textit{K}}$ eines Zahlkörpers $\textit{K}$ (als Eulerprodukt) ein, zeigen, dass sie eine holomorphe Funktion auf  $\{\textit{s}  \in  \C: \re(\text{s})  >  1  \} $ definiert, und leiten ihre Dirichletreihenentwicklung her. Anschließend beweisen wir, dass $\zeta_{\textit{K}}$ im Falle eines quadratischen Zahlkörpers  $\textit{K}$ als Produkt der Riemannschen Zetafunktion $\zeta$ mit einer gewissen $\textit{L}$-Funktion geschrieben werden kann.
Ist $\textit{K}$ imaginär quadratisch, so erhalten wir auf diese Weise eine handliche Formel für die Anzahl der Darstellungen einer natürlichen Zahl $\textit{n} \in \N$ als \textit{n} = $f(x,y)$, wobei $f$ eine reduzierte quadratische Form $f$ $\in \Z[X,Y]$ mit Diskriminante $\Delta_\textit{K}$ ist.

\end{abstract}

\invisiblesection{}

\begin{defn}[Riemannsche Zetafunktion]

Die \textit{Riemannsche Zetafunktion} ist definiert als  $$\zeta \textit{(s)}= \sum_{n = 1}^\infty  \frac{1}{n^s} =  \prod_{\textit{p}} \frac{1}{1-\textit{p}^\textit{-s}} $$ mit  $$ \zeta : \{\textit{s} \in \C: \re(\textit{s}) > 1\}\rightarrow \C$$ \\
Dabei ist $\prod_{\textit{p}} \frac{1}{1-\textit{p}^\textit{-s}}$ die Definition als Eulerprodukt über Primzahlen und $\sum_{n = 1}^\infty  \frac{1}{n^s} $ die Definition als Dirichletreihe.

\end{defn}


\begin{defn}[Zahlkörper]

Ein \textit{(algebraischer) Zahlkörper} ist eine endliche Körpererweiterung der rationalen Zahlen $\Q$.
Sei im Folgenden $\textit{K}$ ein Zahlkörper. Man nennt einen Zahlkörper \textit{quadratisch}, wenn er die Grad 2 über $\Q$ hat. 

\end{defn}

\begin{defn}[Ganzheitsring]

Sei $\textit{K}$ ein Zahlkörper. Dann definiert \\
$O_{K}$ := \textit{K} $\cap$ $\mathbb{A}$ den \textit{Ganzheitsring} des Zahlkörpers $K$, wobei $\mathbb{A}$ als Ring der ganz algebraischen Zahlen definiert ist.
\end{defn}

\begin{defn}[Idealnorm]

Sei $\textit{K}$ ein Zahlkörper mit Ganzheitsring $$\mathrm{N}_{K \mid \Q}(\mathfrak{a}) = |\mathrm{N}(a)|$$ welches ungleich dem Nullideal ist. So ist die \textit{Idealnorm} definiert durch $$\mathrm{N}_{K \mid \Q}(\mathfrak{a}) := [O_{K} : \mathfrak{a} ] = \vert O_{K} / \mathfrak{a} \vert$$

\end{defn}

\begin{note}
(1) Ist $\mathfrak{a}$ ein Hauptideal $\langle a \rangle$, dann ist $$N(\mathfrak{a}) = |N_{K\mid \Q}(a)|$$
\\
(2) Die Norm ist multiplikativ. Es ist N$(\mathfrak{a}\cdot \mathfrak{b}) = $N$(\mathfrak{a})\cdot N(\mathfrak{b})$

\end{note}

\begin{defn}[Dedekindsche Zetafunktion]

Sei $\textit{K}$ ein Zahlkörper. Wir definieren die \textit{Dedekindsche Zetafunktion} \textit{$\zeta_{\textit{K}}$} zum Zahlkörper $\textit{K}$ durch
	
$$\zeta_{\textit{K}} \textit{(s)}:= \prod_{\mathfrak{p}}\frac{1}{1-\mathrm{N}_{K \mid \Q}(\mathfrak{p})^{-s}}$$
wobei sich das Produkt über alle Primideale $\langle0\rangle\subsetneq \mathfrak{p} \subseteq O_{\textit{K}}$ erstreckt. 
\end{defn}

\begin{satz}

Für \textit{K} = $\Q$ stimmt die Dedekindsche Zetafunktion mit der Riemanschen Zetafunktion überein.

\begin{proof}
Es ist $O_\textit{K} = \mathbb{Z}$, also sind die Primideale $\langle0\rangle\subsetneq \mathfrak{p} \subseteq O_{\textit{K}}$ gerade jene Haupideale $\langle \textit{p}\rangle\subseteq\Z$, die von Primzahlen \textit{p} erzeugt werden. Führen wir diesen Gedanken weiter so gilt $N_{\Q\mid\Q}(\langle \textit{p}\rangle) = | \mathbb{Z}/\langle \textit{p}\rangle| = \textit{p}$, also erhalten wir  $$\zeta_{\Q} \textit{(s)}= \prod_{\textit{p}} \frac{1}{1-\textit{p}^\textit{-s}} = \zeta \textit{(s)} $$

\end{proof}

\end{satz}

\begin{defn}[Unendliches Produkt]
Sei $(a_{\nu})\subset \C$. Das \textit{unendliche Produkt} $\prod_{\nu = 1 }^{\infty}(a_{\nu})$ existiert, falls gilt:
\\
\\
(1) Entweder sind alle $a_{\nu}\neq 0$, es existiert der Grenzwert $ a := \lim_{n \to \infty} \prod_{\nu = 1}^{n} a_{\nu}$ und es ist $a \neq 0$
\\
(2) Oder es gibt ein $\nu_{0}$, so dass $a \neq 0$ für alle $\nu \geq \nu_{0}$ ist, und es existiert $a^{*} := \prod_{\nu = \nu_{0}}^{\infty}$ im obigen Sinne. Dann setzen wir $a := a^{*} 
\cdot \prod_{\nu = 1}^{\nu_{0} - 1}a_{\nu}$.

\end{defn}

\begin{note}[Konvergenz von unendlichen Produkten]

Das unendliche Produkt $\prod_{\nu = 1 }^{\infty}(a_{\nu})$ existiere. Dann gilt:
\\
\\
(1) $\prod_{\nu = 1}^{\infty}a_{\nu} = 0$ genau dann, wenn mindestens ein $a_{\nu}$ gleich Null ist.
\\
\\
(2) Die Folge $(a_{\nu})$ ist "1-Folge", das heißt es ist $\lim_{\nu \rightarrow\infty}a_{\nu} = 1$.

\end{note}

\begin{note}[Konstruktion des Logarithmus im Komplexen]

Durch die \\Einschränkung der Exponentialfunktion auf den Streifen $S = \{ z \in \C : $\\$\im(z) \in (-\pi, \pi)\}$ wird sie injektiv und der Satz der Umkehrfunktion ist anwendbar. Daraus folgt:$$\Log(z) := \log(r) + \phi$$ mit $r=|z|$ mit $arg(z)= \phi$. Hierbei bezeichnet log den reellen und Log den Hauptzweig des komplexen Logarithmus.

\end{note}

\begin{note}[Cauchy-Riemann-Differentialgleichug in Polarkoordinaten (CRDFG)]

Durch die Darstellung einer komplexen Zahl als $z = r\cdot e^{i\phi}$ folgt eine besondere Form der CRDFG. Es ist: $$(1) \frac{\delta u}{\delta r} = \frac{1}{r} \frac{\delta v}{\delta \phi}$$ und $$(2)\frac{\delta v}{\delta r} = -\frac{1}{r} \frac{\delta u}{\delta \phi}$$ mit $f := u + iv$.

\end{note}

\begin{folg}[Log ist holomorph]

Mit obiger Gleichung folgt mit $\Log(r\cdot e^{i \phi}) := \log(r) + i \phi$, dass $$\frac{\delta u}{\delta r} =\frac{1}{r} = \frac{1}{r} \frac{\delta v}{\delta \phi} $$ und $$\frac{\delta v}{\delta r} = 0=-\frac{1}{r} \frac{\delta u}{\delta \phi}$$ mit $f = \log(r) + i\phi= u +iv$.

\end{folg}

\begin{note}[Potenzreihe des Logarithmus auf $\C\setminus\{0\}$]

Es gilt $$\Log(1+z) = \sum_{n = 1}{\infty}\frac{(-1)^{(n-1)}}{n}(z-1)^{n}$$ für alle $z \in \C$ mit $|z| < 1$.

\end{note}

\begin{defn}[Normale Konvergenz]

Eine Reihe von Funktionen\\ $f_{1}+f_{2}+f_{3}+....$ $f_{n}: D\rightarrow \C $ für $D \subset \C$ und $n \in \N_{0}$ heißt \textit{normal konvergent in D}, falls es zu jedem Punkt $a \in D$ eine Umgebung $U$ und eine Folge $(M_{n})_{n\geq0}$ nicht negativer reeller Zahlen gibt, so dass gilt $$|f_{n}(z)|\leq M_{n}$$ für alle $z \in U \cap D$, für alle $n \in \N_{0}$ und $\sum_{n = 0}^{\infty}M_{n}$ konvergiert.

\end{defn}

\begin{note}[Identitätssatz]

Seien $f$ und $g$ holomorphe Funktionen auf einer Umgebung $U$ von $z_{0}$ und sei $z_{0}$ ein Häufungspunkt der Koinzidenzmenge\\  $\{ z \in U  |  f(z) = g(z) \}$, dann existiert eine Umgebung $V$ von $z_{0}$ mit $f(z) = g(z)$ für alle $z \in V$.

\end{note}

\begin{lem}\label{lemma12}

Sei $\sum_{n =1}^\infty f_{\textit{n}}$ eine normal konvergente Reihe von holomorphen Funktionen $f_{\textit{n}}: \textit{D}\rightarrow\C $ auf einem Gebiet
$D\subseteq\C$. Dann existiert für jeden Punkt $ z \in \C$ eine offene Umgebung $z \in U \subseteq \C$ so wie eine natürliche Zahl $N \in \N$, sodass $|f_{\textit{n}}(z)| \leq \frac{1}{2}$ für alle $z \in U $ und alle $n \geq N $ gilt. In diesem Fall ist durch $$F_{N}(z) := \sum_{n = N}^\infty \Log( 1 + f_{n}(z))$$ eine holomorphe Funktion $F_{N} : U \rightarrow \C$ definiert. Weiter gilt $$\lim_{m\rightarrow\infty}\prod_{n=1}^m ( 1 + f_{n}(z)) = \prod_{n=1}^{N-1} ( 1 + f_{n}(z))\cdot e^{F_{N}(z)}$$ für alle $z \in U$. Insbesondere stellt das unendliche Produkt$$\prod_{n=1}^\infty ( 1 + f_{n}(z)) := \lim_{m\rightarrow\infty}\prod_{n=1}^m ( 1 + f_{n}(z))$$ eine holomorphe Funktion auf D dar.

\begin{proof}\renewcommand{\qedsymbol}{}
Für den Beweis benötigen wir zunächst zwei Hilfslemmata. 
\end{proof}

\begin{hilfslem}\label{hilfslemma1}

Aus $|f_n(z)|\leq \frac{1}{2}$ folgt $1+f_n(z) \in \C\backslash\R_{\leq 0}$. 

\begin{proof}
Es ist $|f_n(z)|\geq 0$, daher ist die Anwendung der Wurzelfunktion und das Quadrieren eine Äquivalenzumformung. So folgt mit der Definition des komplexen Betrags direkt:
$$|(f_n(z))|^2=\sqrt{f_n(z)\cdot {\iota (f_n(z))}}^2=\sqrt{f_n(z)}^2\cdot\sqrt{{\iota (f_n(z))}}^2= f_n(z)\cdot{\iota (f_n(z))}$$und wegen der Positiven Definitheit des Betrags gilt bereits, das jeweils beide Ausdrücke größer oder gleich Null sind. Die Null wird aber durch die Addition mit der 1 ausgeschlossen. So folgt $1+(f_n(z) \in \C \backslash \R_{\leq 0}$
\\
\\
Insbesondere:
\\
Es ist $\Log( 1 + f_{n}(z))$ für alle $z \in U $ und $n \geq N$ definiert, da aus der obiger Betrachtung hervorgeht, dass $1+f_n(z)$ für alle $z$ im Definitionsbereich von $\Log$ enthalten ist.
\end{proof}
\end{hilfslem}

\begin{hilfslem}\label{hilfslemma2}

Abschätzung $ |Log(1+z)|\leq 2|z|$ für alle $z \in \C $ mit $ |z| \leq \frac{1}{2}$

\begin{proof}

$$|\Log(1+z)|= |\sum_{n = 1}^\infty (-1)^{(n-1)}\frac{z^{n}}{n}|\leq|z|\sum_{n = 0}^\infty\frac{|z|^{n}}{n + 1}$$
$$< |z|\cdot (\frac{1}{1}+\frac{1}{2\cdot 2}+\frac{1}{3 \cdot2^2}+...+\frac{1}{(n+1)\cdot2^(n)}+....) = 2\cdot|z| $$
\\
\\
wobei bei dem letzten Gleichheitszeichen der Wert der geometrischen Reihe benutzt wurde. Nämlich $$\sum_{n=0}^{\infty}\frac{1}{(n+1)\cdot2^{(n)}} < \sum_{n=0}^{\infty}\frac{1}{2^{(n)}} = \sum_{n=0}^{\infty}(\frac{1}{2})^n = \frac{1}{1-(\frac{1}{2})} = 2 $$

\end{proof}
\end{hilfslem}

\begin{proof}[Beweis zu Lemma \ref{lemma12}]
\noindent Aus Hilfslemma \ref{hilfslemma1} folgt $(1 + f_{n}(z)) \in \C \setminus \R_{\leq 0 }$. \\
 Zeige weiter $F_{N}(z) := \sum_{n = N}^\infty \Log( 1 + f_{n}(z))$ ist normal konvergent auf U. Dies folgt daraus, dass $| Log( 1 + z) | \leq 2|z| $ für $z \in \C$ mit $|z| \leq \frac{1}{2}$ nach Hilfslemma \ref{hilfslemma2}  und der Tatsache, dass mit der Abschätzung von $|Log(1+z)| \leq 2|z|$ mit $|f_n(z)|\leq \frac{1}{2}$ $F_{N}(z)$ genau der Definition der normalen Konvergenz entspricht.
Insbesondere stellen $F_{N}$ und $e^{F_{N}}$ holomorphe Funktionen dar, da Holomorphie unter Bildung von Summen, Produkten und Verkettungen erhalten bleibt. Für beliebige $z \in \C $ und $m\leq N$ gilt, dass 
$$\prod_{n = 1 }^{m}( 1 + f_{n}(z)) = \prod_{n = 1}^{N - 1}( 1 + f_{n}(z)) \cdot \prod_{n = N}^{m}( 1 + f_{n}(z))= \prod_{n = 1}^{N - 1}( 1 + f_{n}(z)) \cdot \prod_{n = N}^{m}(e^{Log( 1 + f_{n}(z)}) $$ $$= \prod_{n = 1}^{N - 1}( 1 + f_{n}(z)) \cdot \prod_{n = N}^{m}(e^{\sum_{n = N}^{m}Log( 1 + f_{n}(z)})$$
wobei hier benutzt wurde, dass $e^{F_{N}}$ auf der bereits oben angesprochenen Einschränkung konform mit dem komplexen Logarithmus agiert. Mit der Stetigkeit der Exponentialfunktion folgt $$\lim_{m\rightarrow\infty}\prod_{n=1}^m ( 1 + f_{n}(z)) =\prod_{n = 1}^{N - 1}(1 + f_{n}(z))\cdot e^{F_{N}(z)}$$ für alle $z \in U$ gilt. Durch die Holomorphie von $1+ f_{n}$ für $1 \leq n \leq N - 1$ und $e^{F_{N}}$ und dem Umstand, dass endliche Produkte holomorpher Funktionen wiederum holomorph sind, stellt $$\prod_{n =1}^{\infty}(1 +f_n (z))$$ als unendliches Produkt eine holomorphe Funktion - nicht nur für alle $z_0$ auf $U$, sondern aufgrund der Offenheit der Umgebung $U$ auf ganz $D$ - dar.
\\
\\
\end{proof}
\end{lem}

\begin{prop}

Sei $\textit{K}$ ein Zahlkörper. Dann konvergiert das unendliche Produkt $$\zeta_{\textit{K}} \textit{(s)}= \prod_{\mathfrak{p}}\frac{1}{1-\mathrm{N}_{K \mid \Q}(\mathfrak{p})^{-s}}$$ für $s \in \C $ mit $ \re(s) > 1$. Des weiteren stellt $\zeta_{K}(s)\colon \{ s \in \C \colon \re(s)>1\} \rightarrow \C$ eine holomorphe Funktion dar.

\begin{proof}\label{proofprop14}
Das unendliche Produkt
$$\zeta*_{\textit{K}} \textit{(s)}= \prod_{\mathfrak{p}}(1-\mathrm{N}_{K \mid \Q}(\mathfrak{p})^{-s})$$ für $s \in \C $ mit $\re(s) > 1$ konvergiert, beziehungsweise ist auf diesem Gebiet holomorph.
Zunächst ist es wichtig zu sehen, dass keiner der Faktoren $1-\mathrm{N}_{K \mid \Q}(\mathfrak{p})^{-s}$ eine Nullstelle in der Menge $s \in \C $ mit $\re(s) > 1$ besitzt.
\\
Dies folgt schnell über: 
\\
Angenommen $1- N_{K|\Q}(\mathfrak{p})^{(-s)}$ hat eine Nullstelle in der Menge  $s \in \C $ mit $ \re(s) > 1$, so folgt $1 = N_{K|\Q}(\mathfrak{p})^{(-s)}$. Das bedeutet nach der Norm |$O_K /\mathfrak{p}| = 1$, d. h. $\mathfrak{p} = O_K$. Dann ist aber $\mathfrak{p}$ nicht echt in $O_K$ und daher kein Primideal. \\
\\
Trifft die Behauptung auf $\zeta*_{K}$ zu, so gilt zusätzlich $\zeta*_{K} \neq 0$ für alle  $s \in \C $ mit $ \re(s) > 1$. Denn wäre  $\zeta*_{K} = 0$, so wäre nach oben stehendem Satz mindestens ein Faktor gleich Null. Dies haben wir aber eben ausgeschlossen. So ist auch  $\zeta*_{K}^{-1}$ holomorph auf diesem Gebiet. \\Es genügt nun zu zeigen, dass 
$\zeta*_{K}^{-1}$ = $\zeta_{K}$. Es ist $$\zeta*_{K}^{-1}=(\prod_{\mathfrak{p}}(1 - N_{N\mid\Q}(\mathfrak{p})^{(-s)}))^{-1} $$ $$=\prod_{\mathfrak{p}}(1 - N_{N\mid\Q}(\mathfrak{p})^{(-s)})^{-1} = \prod_{\mathfrak{p}}\frac{1}{1 - N_{N\mid\Q}(\mathfrak{p})^{(-s)}}=\zeta_{K}$$
Damit folgt $\zeta_{K}$ ist holomorph auf $\{ s \in \C \colon \re(s) > 1\}$, falls wir zeigen können, dass $\zeta*_K$ holomorph ist. Der Nachweis der normalen Konvergenz der Funktionenreihe $$\sum_{\mathfrak{p}}N_{N\mid\Q}(\mathfrak{p})^{-s}$$ auf $\{ s \in \C \colon Re(s) > 1\}$ folgt aus $$\sum_{\mathfrak{p}}|e^{Log(N_{K\mid \Q}(\mathfrak{p})^{(-s)}}| = \sum_{\mathfrak{p}}|e^{-s\cdot Log(N_{K\mid \Q}(\mathfrak{p})}|$$
 und unter der Verwendung von Folgendem: \\Es ist mit $z = x + iy$ $$|e^z| = e^x $$ da $$|e^z|=|e^{x+iy}|=|e^{x}\cdot e^{iy}|= |e^{x}|\cdot |e^{iy}| = e^{x}\cdot |e^{iy}| = e^x \cdot |\cos(y)+i \cdot \sin(y)| $$ $$= e^x\cdot\sqrt{\cos^2(x)+\sin^2(x)} =e^x\cdot 1$$
und mithilfe der folgenden Betrachtung: \\Jedes Primideal $\langle0\rangle\subsetneq \mathfrak{p} \subseteq O_{\textit{K}}$ in $\mathfrak{p} \in O_K$ enthält eine Primzahl $p$ und $\mathfrak{p}$ tritt in diesem Fall in der Primidealzerlegung (siehe Vortrag 5) des Hauptideals $\langle p \rangle=p\cdot O_K$ auf. Somit gibt es zu jeder Primzahl $p$ nur endlich viele Primideale $\mathfrak{p}\subset O_K$, die $p$ enthalten. Präziser gilt: Für fast alle Primzahlen $p$ gibt es höchstens $[K\colon \Q]$ verschiedene Primideale $\mathfrak{p}\subset O_K$, die $p$ enthalten. 
\\
So folgt:

$$\sum_{\mathfrak{p},p\in \mathfrak{p},p>C}(N_{K\mid \Q}(\mathfrak{p}))^{-\re(s)} \leq [K\colon\Q]\cdot \sum_{p>C}p^{-\re(s)} $$Den letzten Ausdruck kann man mithilfe der Geometrischen Reihe abschätzen.
\end{proof}
\end{prop}


\begin{defn}[Dirichletreihe]

Eine \textit{(formale) Dirichletreihe} ist eine Reihe der Form $$\mathbb{D}(s) = \sum_{n =1}^{\infty}\frac{a_n}{n^s} $$ wobei $(a_n)_{n \in \N}$ eine beliebige Folge komplexer Zahlen beschreibt.

\end{defn}

\begin{prop}

Sei $\textit{K}$ ein Zahlkörper. Dann gilt für alle $s \in \C $ mit $ \re(s)>1$, dass$$\zeta_{K}(s) = \sum_{\mathfrak{J}}\frac{1}{N_{K\mid\Q}(\mathfrak{J})^s}=\sum_{n=1}^{\infty}\frac{a_n}{n^s}$$
wobei sich die Summation in $\sum_{\mathfrak{J}}$ über alle Ideale $\langle0\rangle\subsetneq \mathfrak{p} \subseteq O_{\textit{K}}$ erstreckt und $a_n$ für $n \in \N$ die Anzahl der Ideale $\mathfrak{J} \subseteq O_K$ mit $N_{K\mid \Q}(\mathfrak{J}) = n$ bezeichnet. Insbesondere gilt: $$\zeta_{K(s)}= \sum_{n=1}^{\infty}\frac{1}{n^s}$$

\begin{proof}

Zur Vereinfachung, führt man eine beliebige Nummerierung auf der Menge aller Primideale $\langle0\rangle\subsetneq \mathfrak{p} \subseteq O_{\textit{K}}$ ein. Diese hat die Form $$\{\langle0\rangle\subsetneq \mathfrak{p} \subseteq O_{\textit{K}}\}=\{\mathfrak{p}_1,\mathfrak{p}_2,\mathfrak{p}_3,...\}$$Die Abzählbarkeit dieser Menge folgt aus: 
\\
Jedes Primideal $\langle0\rangle\subsetneq \mathfrak{p} \subseteq O_{K}$ in $O_K$ enthält eine Primzahl $p$. Die Primideale, die in $O_K$ sind und $p$ enthalten, sind genau die Primideale, die in der Faktorisierung des Hauptideals $pO_K$ auftreten (nach Vortrag 5). In der Faktorisierung von $pO_K$ treten aber nur endlich viele Primideale auf, das heißt zu jeder Primzahl $p$ gibt es nur endlich viele Primideale, die $p$ enthalten. Somit besteht die Menge der Primzahlen, die $p$ enthalten, aus einer abzählbaren Vereinigung von abzählbaren Mengen - nämlich die Vereinigung über alle Primzahlen und die Vereinigung aller Primideale, die $p$ enthalten. Abzählbare Vereinigungen abzählbarer Mengen sind wiederum abzählbare Mengen.
\\
Sei weiter $s \in \C $ mit $ \re(s) > 1$ beliebig, dann ist mit $N_{K\mid \Q}\geq 2$ für jedes $m \in \N$,  siehe Beweis Proposition \ref{proofprop14}. So folgt:$$\prod_{m=1}^{n}\frac{1}{1-N_{K\mid \Q}(\mathfrak{p}_m)^{(-s)}} =\prod_{m=1}^{n}\sum_{k=0}^{\infty}(N_{K\mid \Q}(\mathfrak{p}_m))^{-k\cdot(s)}=\prod_{m=1}^{n}\sum_{k=0}^{\infty}(N_{K\mid \Q}(\mathfrak{p}_m^k))^{(-s)}$$
Daraus resultiert mit Anwendung der Geometrischen Reihe auf \\
$(N_{K\mid \Q}(\mathfrak{p}_m^k))^{(-s)} < 1$, dass das Produkt absolut konvergiert, da es Produkt absolut konvergenter Reihen ist. Mithilfe des Umordnungssatzes für absolut konvergente Reihen und der Mulitplikativität der Norm spricht dann:
$$\prod_{m=1}^{n}\sum_{k=0}^{\infty}(N_{K\mid \Q}(\mathfrak{p}_m^k))^{(-s)}=1+\sum_{j=1}^{n}\sum N_{K\mid \Q}(\mathfrak{p}_{m_1}^{\alpha_1})^{(-s)}\cdot\cdot\cdot N_{K\mid \Q}(\mathfrak{p}_{m_j}^{\alpha_j})^{(-s)}$$ $$=1+\sum_{j=1}^{n}\sum N_{K\mid \Q}(\mathfrak{p}_{m_1}^{\alpha_1}\cdot\cdot\cdot\mathfrak{p}_{m_j}^{\alpha_j})^{(-s)}$$
wobei sich die unbeschriftete Summe über alle Teilmengen $\{m_{1},...,m_{j}\}\subseteq \{ 1,...,n\}$ und alle $(\alpha_1,...,\alpha_j)$ erstreckt. Mit $n \rightarrow \infty$ folgt mit der vorherigen Proposition, dass $$1+\sum_{j=1}^{\infty}\sum N_{K\mid \Q}(\mathfrak{p}_{m_1}^{\alpha_1}\cdot\cdot\cdot\mathfrak{p}_{m_j}^{\alpha_j})^{(-s)}$$
Aufgrund der eindeutigen Primidealzerlegung in $O_K$ und dass $\N_K(O_K)^{-s}=1$ tritt jedes Ideal $\langle0\rangle\subsetneq \mathfrak{p} \subseteq O_{\textit{K}}$ genau einmal als $\mathfrak{p}_{m_1}^{\alpha_1}\cdot\cdot\cdot\mathfrak{p}_{m_j}^{\alpha_j}$ genau einmal auf. Wieder aufgrund der Umordnung folgt daraus $$\zeta_{K}(s) = \sum_{\mathfrak{J}}\frac{1}{(N_{K\mid\Q}(\mathfrak{J}))^s}=\sum_{n=1}^{\infty}\frac{a_n}{n^s}$$ und im Falle $\textit{K}$ = $\Q$ gibt es für jede natürliche Zahl genau ein Ideal mit Norm $n$, nämlich $\langle n\rangle=n\Z\subseteq \Z$, daher gilt $a_n=1$ für alle $n\in \N$. So folgt insbesondere: $$\zeta_{K(s)}= \sum_{n=1}^{\infty}\frac{1}{n^s}$$

\end{proof}
\end{prop}

\begin{note}

Der aus Vortrag 8 existierende Gruppenhomomorphismus ist als Funktion aufgefasst $\chi_k \colon \N \rightarrow \{-1,0,1 \}$, wobei man für $n \in \N$ schreibt, dass $$\chi_K=\begin{cases}0 \ falls \ \ggT(n,\Delta_K) >1,\\\chi_{K}(\bar{n})\ falls\ \ggT(n,\Delta_K) =1\end{cases}$$

\end{note}

\begin{thm}

Sei $K=\Q(\sqrt{d})$ ein quadratischer Zahlkörper. Dann definiert das Eulerprodukt $$L(s,\chi_K) := \prod_{p}\frac{1}{1-\chi_K(p)p^{(-s)}} $$ eine holomorphe Funktion $L(s,\chi_K) \colon \{s \in \C \colon \re(s) > 1\} \rightarrow \C$ und es gilt $$\zeta_K(s) = \zeta(s)\cdot L(s,\chi_K) $$ für alle $s \in \C $ mit $ \re(s) > 1$. Man nennt $L(\cdot,\chi_k)$ die L-Funktion zum Gruppenhomomorphismus $\chi_K$.

\begin{proof}

Wenn wir zeigen können, dass $$\prod_{\mathfrak{p},p \in \mathfrak{p}}\frac{1}{(1- N_{K\mid \Q}(\mathfrak{p}))^{(-s)}}= \frac{1}{1-p^{-s}}\cdot\frac{1}{1-\chi_k(p)p^{-s}}$$ für jede Primzahl $p$ gilt, dann folgt daraus, dass $$L(s,\chi_K)= \zeta_K(s)\cdot \zeta(s)^{-1}$$ für alle $s \in \C $ mit $ \re(s)>1$ gilt, da $\zeta(s) \neq 0$. Dadurch ist $$L(\cdot,\chi_K)= \zeta_K\cdot \zeta^{-1}\colon \{s \in \C \colon \re(s)>1\}\rightarrow \C$$ eine holomorphe Funktion.
\\
\\
\textit{Fall 1.}
$O_K= \Z[\sqrt{-d}]$ \\
Betrachte also $d\equiv2,3 ($mod $ 4)$
Die Primideale $\mathfrak{p} \subseteq O_K$, die $p$ enthalten, entsprechen genau den Primidealen im Ring $$O_K /\langle p\rangle \cong \Z[X] /\langle X^2-d,p\rangle \cong (\Z/p\Z)[X]/\langle X^2-\bar{d}\rangle$$  Dies folgt aus Vortrag 3, Beweis Proposition 1, aus dem Isomorphiesatz.
\\
\\
\textit{Fall 1.1.}
Angenommen $p \mid \Delta_K$. Das bedeutet im Fall $d \equiv2,3 ($mod $ 4)$, dass entweder $p = 2$ oder $p \mid d$. Im ersten Fall ist $X^2-\bar{d} = (X-\bar{d})^2 \in (\Z/2\Z)[X]$.
Im zweiten Fall ist $X^2-\bar{d} = X^2 \in (\Z/p\Z)[X]$, also hat das Polynom $X^2-\bar{d} \\ \in (\Z/p\Z)$ auf jeden Fall eine doppelte Nullstelle in $\Z/p\Z$. Vermöge des Isomorphismus von oben bedeutet das einerseits, dass $\langle p\rangle \subseteq O_K$ kein Primideal ist und andererseits, dass es ein eindeutiges Primideal $\mathfrak{p} \subseteq O_K$ mit $p \in \mathfrak{p}$ gibt. Dies folgt daraus, dass eine Primzahl genau dann prim in $O_K$  ist, wenn das Polynom keine Nullstellen in $(\Z/p\Z)[X]$ hat. Des Weiteren ist $\Z/p\Z$ ein Körper. Das einzige maximale und damit prime Ideal ist das Nullideal, welches $p$ enthält.
\\
\\
In Anbetracht der eindeutigen Primfaktorzerlegung kann das nur bedeuten, dass $\langle p\rangle =\mathfrak{p}^k$ für ein $k\geq 2$ gilt. Da aber $p^2=N_{K\mid \Q}(\langle p\rangle)=N_{K\mid \Q}(\mathfrak{p}^k) =N_{K\mid \Q}(\mathfrak{p})^k $ gilt, folgt aus der eindeutigen Primzfaktorzerlegung in $\Z$, dass $k = 2$ und $N_{K\mid\Q}(\mathfrak{p}) =p$. So ist die linke Seite gegeben durch $\frac{1}{1-p^{-s}}$ und die rechte Seite $$\frac{1}{1-p^{-s}}\cdot \frac{1}{1-0\cdot p^{-s}}=\frac{1}{1-p^{-s}}$$also die Gleichheit beider Seiten.
\\
\\
\textit{Fall 1.2.}
Angenommen $p \nmid \Delta_K$ und $\chi_K(\bar{p})=(\frac{\Delta_K}{p})= 1$. Da $\Delta_K = 4d$, folgt daraus, dass $$(\frac{d}{p})=(\frac{2}{p})^2\cdot (\frac{d}{p})=(\frac{4d}{p})=(\frac{\Delta_K}{p})=1$$Dann liefert die Definition des Legendre Symbols, dass $\bar{d} \in (\Z/p\Z)^\times$ ein Quadrat ist, sagen wir $\bar{d}=\bar{y}^2$ für ein $y \in \Z$,  $p \nmid y$. Das ergibt, dass $$X^2-\bar{d}=(X-\bar{y})(X+\bar{y})$$ in $(\Z/p\Z)[X] $ gilt, also dass $X^2-\bar{d}$ zwei verschiedene Nullstellen in $\Z/p\Z$ hat. Das bedeutet, dass es genau zwei verschiedene Primideale in $O_K$ gibt, die $p$ enthalten, und wie in Fall 1.1, dass diese Norm $p$ haben. Somit ist die linke Seite gegeben durch $(\frac{1}{1-p^{-s}})^2$, die rechte Seite durch $$\frac{1}{1-p^{-s}}\cdot\frac{1}{1-1\cdot p^{-s}}=(\frac{1}{1-p^{-s}})^2$$ und damit wiederum gleich.
\\
\\
\textit{Fall 1.3.}
Angenommen $p \nmid \Delta_K$ und $\chi_K(\bar{p})=(\frac{\Delta_K}{p})= -1$. Wie in Fall 1.2 folgt $(\frac{d}{p})=-1$, also dass $\bar{d} \in (\Z/p\Z)^\times$ kein Quadrat ist. Das bedeutet aber, dass das Polynom $X^2-\bar{d} \in (\Z/p\Z)[X] $ keine Nullstelle in $\Z/p\Z$ besitzt. Bei einem Polynom von Grad 2 ist das wiederum äquivalent dazu, dass es keine Nullstellen in $\Z/p\Z$ besitzt. Damit ist das Polynom irreduzibel. Daraus folgt aber $(\Z/p\Z)[X]/\langle X^{2}-\bar{d}\rangle$ ist ein Körper. Also trifft wegen des Isomorphismus Selbiges auf $O_K/\langle p\rangle$ zu. Das macht das von $p$ erzeugte Ideal in $O_K$ zu einem maximalen und insbesondere zu einem Primideal. Somit haben wir wieder für die linke Seite $\frac{1}{1-(p^{2})}^{-s}=\frac{1}{1-(p)}^{-2s}$ und die rechte Seite stimmt damit überein durch $$\frac{1}{1-(p^{-s})}\cdot\frac{1}{(1-(-1)1p^{-s})}=\frac{1}{(1-p^{-s}(1+p^{-s})}=\frac{1}{1-p^{-2s}}$$
\\
\\
\textit{Fall 2.}
$O_K= \Z[\frac{1+\sqrt{d}}{2}]$ \\
Betrachte also $d\equiv 1 ($mod $ 4)$ 
\\
\\
\textit{Fall 2.1.}
Angenommen $p \mid \Delta_K$. Im Falle $d\equiv 1 ($mod $ 4)$ bedeutet das $p\mid d$ da  $\Delta_K=d$. Das folgt aus dem Bilden der Diskriminante aus dem Polynom $4\cdot(X^2-X+\frac{1-d}{4})$. So folgt direkt $2(X-1)^{2}-d$ hat zwei Nullstellen in $\Z\backslash p\Z[X]$ und wie in Fall 1.1 folgt Gleichheit der beiden Seiten.
\\
\\
\textit{Fall 2.2.}
Angenommen $p\nmid \Delta_K$ und $\chi_K(\bar{p})=(\frac{\Delta_K}{p})=1$. So folgt direkt $(\frac{d}{p})=1$ mit Definition des Legendre Symbol, dass $\bar{d} \in (\Z\backslash p\Z)^\times$ ein Quadrat ist und somit hat $(2x-1)^2-d$ zwei Nullstellen in $\Z\backslash p\Z$. Der Rest folgt analog dem Beweis von Fall 1.2.
\\
\\
\textit{Fall 2.3}
Angenommen $p\nmid \Delta_K$ und $\chi_K(\bar{p})=(\frac{\Delta_K}{p})=-1$. So ist $\bar{d}$ wie in Fall 1.3 kein Quadrat und hat in $\Z\backslash p\Z$ keine Nullstellen. Der Beweis schließt analog wie in Fall 1.3.


\end{proof}
\end{thm}

\begin{thm}\label{theo16}

Sei $K=\Q(\sqrt{d})$ ein imaginär quadratischer Zahlkörper. Für jede natürliche Zahl $n \in \N$ bezeichnen wir mit $r_k(n)$ die Anzahl der Paare $(x,y) \in \Z^2$, sodass $n = f(x,y)$ für eine reduzierte quadratische Form $f \in \Z[X,Y]$ mit Diskriminante $\Delta_K$ gilt. Dann ist $$r_k(b) =^{(1)} |O_K^\times|\cdot a_n =^{(2)} |O_K^\times|\cdot \sum_{d|n}\chi_K(n)$$wobei $(a_n)_{n \in \N}$ die Folge der Koeffizienten der Dirichletreihe zu $\zeta_K$ ist, und $$|O_K^\times|=\begin{cases}4, falls \ d = -1\\6, falls \ d = -3\\2, sonst\end{cases}$$ gilt.
\\
\begin{proof}\renewcommand{\qedsymbol}{}
Die Aussage über $|O_K^\times|$ wurde in Vortrag 3 bewiesen. Somit verbleibt es die Gleichheitszeichen (1) und (2) zu zeigen. Beginne mit (2). Es ist $$L(s,\chi_K)=\sum_{n=1}^{\infty}\frac{\chi_K(n)}{n^s}$$ für alle $s \in \C $ mit $ \re(s) > 1$. Dies folgt aus dem folgenden Hilfslemma \ref{hilfslemma3}.
 
 \end{proof}
 \end{thm}
 
 \begin{hilfslem}\label{hilfslemma3}
Sei $f$ eine multiplikative Funktion, sodass die $\sum f_n(n)$ absolut konvergiert. So kann der Wert der Reihe als ein absolut konvergentes Produkt ausgedrückt werden. Im Spezialfall der strikten Multiplikativität von $f$ erhält man $\sum_{n=1}^{\infty}f_n=\prod_{\mathfrak{p}}$.

\begin{proof}

Sei $$P(x)= \prod_{p\leq x}\{1+f(p)+f(p^2)+...\}$$welches sich über alle Primzahlen $p$ erstreckt. Da es ein Produkt einer endlichen Anzahl von absolut konvergenter Reihen ist, ist es erlaubt die Reihen zu multiplizieren und gegebenenfalls umzuordnen ohne den Wert zu verändern. Hierbei wählt man die Form
$$f(p_1^{a_1})\cdot f(p_2^{a_2})\cdot f(p_3^{a_3})\cdot....\cdot f(p_{r}^{a_r})=f(p_{1}^{a_1}\cdot p_{2}^{a_2}\cdot p_{3}^{a_3}\cdot ....\cdot p_{r}^{a_r})$$Dies folgt aus der Multiplikativität von $f$. Benutzt man nun die Eindeutigkeit der Primfaktorzerlegung, so folgt, dass $P(x)=\sum_{n \in A}f(n)$ wobei A aus allen $n$ besteht mit Primfaktoren, welche kleiner oder gleich $x$  sind. Daraus folgt $$\sum_{n=1}^{\infty}f(n)-P(x)= \sum_{n \in B}f(n)$$wobei B die Menge derjenigen $n$ ist, die mindestens einen Primfaktor strikt größer $x$ besitzen. Hieraus resultiert$$|\sum_{n=1}^{\infty}f(n)-P(x)|\leq\sum_{n \in B}|f(n)|\leq\sum_{n > x}|f(n)|$$ Läuft jetzt $x\to\infty$, so geht der ganz rechte Ausdruck gegen Null, da $\sum|f(n)|$ konvergent ist. So konvergiert $P(x)$ gegen $\sum f(n)$ für $x \to \infty$. Hier ist es noch wichtig zu sehen, dass das unendliche Produkt der Form $\prod1+a_n$ absolut konvergiert, wenn $\sum a_n$ absolut konvergiert. In diesem Fall folgt:$$\sum_{p\leq x}|f(p)+f(p^2)+...| \leq \sum_{p \leq x}(|f(p)|+|f(p^2)|+...))\leq \sum_{n=2}^{\infty}|f(n)|$$Da alle Partialsummen beschränkt sind, folgt, dass $\sum_{p}|f(p)+f(p^2)+...|$ konvergiert und daraus folgt die absolute Konvergenz des oben genannten unendlichen Produktes.
\\
Im strikt multiplikativem Fall folgt sogar $f(p^n)=f(p)^n$ und jede Reihe im unendlichen Produkt (von oben) ist eine konvergente geometrische Reihe mit dem Wert $(1-f(p))^{-1}$.
\\
\\
In diesem Fall  erhält man $$\sum_{n=1}^{\infty}\frac{f(n)}{n^s}=\prod_{p}\frac{1}{1-f(p)p^{-1}}$$ für $\sum f(n)n^{-1}$ absolut konvergent. Nun setze  $f(n)=\chi_K(n)$ und man erhält $$L(s,\chi)=\sum_{n=1}^{\infty}\frac{\chi(n)}{n^s}=\prod_{p}\frac{1}{1-\chi(p)p^{-s}}$$und damit die Behauptung.\\
Dafür ist die Beobachtung notwendig, dass $\chi_K \colon \rightarrow \{-1,0,1\}$ strikt multiplikativ ist. Das heißt es gilt $\chi(mn)=\chi(m)\cdot\chi(n)$ für beliebige $m,n \in \N$. \\
\\
Die Fallunterscheidung nach o.B.d.A. $\ggT(n,\Delta_K)>1$ hat direkt zur Folge, dass $\ggT(nm,\Delta_K)>1$ und daraus resultiert $\chi_K(nm)=0=0\cdot \chi(m)= \chi(m)\cdot \chi(n)$.
Nun bleibt $\ggT(n,\Delta_K)=1$ so wie $\ggT(m,\Delta_K)=1$. So folgt nach der Definition der Abbildung für beliebige natürliche Zahlen als Gruppenhomomorphismus $$\chi(mn)=\chi(\bar{n})\cdot \chi(\bar{m})= \chi(m)\cdot \chi(n)$$

 \end{proof}
 \end{hilfslem}

\begin{proof}[Fortsetzung Beweis zu Theorem \ref{theo16}]\renewcommand{\qedsymbol}{}
 Das führt zur Behauptung $$\zeta(s)\cdot L(s,\chi_K) = \sum_{n=1}^{\infty}(\sum_{d|n}\chi_K(d))\frac{1}{n^s}$$ für alle $s \in \C $ mit $ \re(s) >1$, wobei die Summe $\sum_{d|n}$ über alle positiven Teiler $d \in \N$ von $n$ durchläuft, da die beiden Dirichletreihen $$\zeta(s) = \sum_{m=1}^{\infty}\frac{1}{m^s} $$ und $$L(s,\chi_K)= \sum_{d=1}^{\infty}\frac{\chi_K(d)}{d^s}$$ für alle $s \in \C $ mit $ \re(s) >1$ absolut konvergieren. Dies folgt für beide direkt nach Anwendung der Geometrischen Reihe.
 \\
 So konvergiert auch das Produkt aus beiden mit der Form $$\zeta(s) \cdot L(s,\chi_K) = \sum_{m=1}^{\infty} \sum_{d=1}^{\infty}\frac{\chi_K(d)}{(md)^s}$$ absolut.
Dadurch ist die Umordnung der Summation beliebig veränderbar zu $$\zeta(s) \cdot L(s,\chi_K)=\sum_{m=1}^{\infty} \sum_{d=1}^{\infty}\frac{\chi_K(d)}{(md)^s}= \sum_{n=1}^{\infty}(\sum_{d \mid n}\chi_K(d))\frac{1}{n^s}$$
Im letzten Schritt nutzt es, dass $n \cdot m = n$ wiederum in $\N$ durch ein beliebiges $n \in \N$ dargestellt werden kann.\\
\\Um nun auch (1) zu beweisen, muss zunächst folgendes Lemma betrachtet werden.
 
 \end{proof}


 
\begin{mot}
Die Identität  $\zeta_K(s) = \zeta(s)\cdot L(s,\chi_K) $ aus vorigem Theorem liefert nun eine also eine Identität von absolut konvergenten Dirichletreihen für $s \in \C $ mit $ \re(s) > 1$ $$ \sum_{n=1}^{\infty}\frac{a_n}{n^s}= \sum_{n=1}^{\infty}(\sum_{d|n}\chi_K(d))\frac{1}{n^s}$$
wobei $a_n = \sum_{d|n}\chi_K(d)$ für beliebige $n \in \N$ aus dem nächsten Lemma folgt.
 \end{mot}
 
 \begin{lem}
 
Sei $N \in \N_0$ und $f \colon \{s \in \C \colon \re(s)>N\}\rightarrow \C$ eine holomorphe Funktion. Falls f für alle $s \in \C$ mit $ \re(s) >N$ durch eine absolut konvergente Dirchletreihe $$f(s) = \sum_{n=1}^{\infty}\frac{b_n}{n^s}$$gegeben ist, dann ist die Koeffizientenfolge $(b_n)_{n \in \N}$ eindeutig durch $f$ festgelegt. Dies ist äquivalent zur Einzigartigkeit der Darstellung von $f$ als absolut konvergente Dirichletreihe, sofern die Darstellung überhaupt existiert.
 \\
\begin{proof}
 Es bietet sich ein induktives Vorgehen an.
Zunächst steht die Behauptung $$b_1= \lim_{k \to \infty} f(k)$$also, dass $b_1$ Grenzwert der Folge $$(f(N+1),f(N+2),f(N+3),.....)$$ ist. Denn dadurch ist $b_1$ eindeutig durch $f$ festgelegt. Dies gilt, da $$f(s)=b_1+\sum_{n=1}^{\infty} \frac{b_n}{n^s}\xrightarrow{s \to \infty} b_1 + 0 = b_1$$
\\ 
Dadurch ist $b_1,...,b_m$ für ein $m \in \N$ eindeutig durch $f$ festgelegt. Ersetze $f(s)$ durch $f(s)-\sum_{n=1}^{m}\frac{b_n}{n^s}$. Daraus folgt die Annahme, dass $b_1=....=b_m=0$ gilt. Mit selbigem Argument wie oben folgt $$b_{m+1}=\lim_{k\to \infty}(m+1)^k f(k)$$ also ist auch $b_{m+1}$ eindeutig durch $f$ festgelegt.
\\
Damit ist erreicht, was in der Motivation gefordert war.

\end{proof}

\begin{proof}[Fortsetzung Beweis zu Theorem \ref{theo16}]
Nun folgt der Beweis zur Rechtfertigung des ersten Gleichheitszeichen des Theorems \ref{theo16}. Zuallererst ist $$\zeta_K(s) = \sum_{C \in Cl_K}\sum_{\mathfrak{J} \in C}\frac{1}{N_{K\mid \Q}(\mathfrak{J})^s}$$wobei sich die Summe $\sum_{\mathfrak{J} \in C}$ über alle Ideale $\langle 0 \rangle \subsetneq \mathfrak{J} \subseteq O_K$ aus der Idealklasse $C \in Cl_K$ erstreckt. Das erste Summenzeichen ist nach Vortrag 7 bereits eine endliche Summe. Man fixiere ein beliebiges Ideals $\langle 0 \rangle \subsetneq \mathfrak{L} \subseteq O_K$ mit \\$\mathfrak{L}\in C^{-1}$, wobei $C^{-1}$ die inverse Idealklasse zu $C$ in $Cl_K$ bezeichnet. Die Behauptung liegt nahe, dass  $$\{Ideale \ \langle 0 \rangle \subsetneq \mathfrak{J} \subseteq O_K \ mit \ \mathfrak{J} \in \C\}\rightarrow(\mathfrak{L}\backslash\{0\})/O_K^{\times}=\{ \beta \cdot O_K^{\times} \colon \beta \in \mathfrak{L}\backslash\{ 0\}\}$$ gegeben durch $$\mathfrak{J}\mapsto \ \dq Erzeuger\ des\ Haupdideals \ \ \mathfrak{J\cdot L} \dq \cdot O_K^{\times}$$ eine wohldefinierte Bijektion darstellt. \\
\\
Zunächst die Wohldefiniertheit:
Es ist $\mathfrak{J\cdot L}\subseteq O_K$ ein Hauptideal mit Namen $\langle \alpha \rangle=\mathfrak{J\cdot L}$, weil $\mathfrak{J\cdot L} \in C\cdot C^{-1} = P_K$, wobei $P_K \in Cl_K$ die Klasse aller gebrochenen Hauptideale, also das neutrale Element bezeichnet. Darüber hinaus ist $\alpha \in \langle \alpha \rangle=\mathfrak{J\cdot L}\subseteq \mathfrak{L}$. Es folgt leicht, dass genau dann $\langle \alpha \rangle=\langle \beta \rangle$ für beliebige $\beta \in \mathfrak{L}$, wenn $\alpha$ und $\beta$ assoziiert sind. Denn sei $\alpha=\beta \cdot y$ mit $y \in O_K^{\times}$, so folgt $\alpha \mid \beta$ und damit unmittelbar $\beta \in \langle \alpha \rangle$ und $\langle \beta \rangle  \subseteq \langle \alpha \rangle$. Dasselbe gilt für $y:= y^{-1}$, da $y$ eine Einheit ist. Also analog für die Vertauschung von $\alpha$ und $\beta$. Die Rückrichtung nutzt, dass $O_K$ ein Hauptidealring ist. Sei weiter $\langle \alpha \rangle=\langle \beta \rangle$. Nach Wahl der Darstellungen für $\beta=\alpha\cdot c$ und $\alpha=\beta\cdot d$ folgt $\alpha=c\cdot d \cdot \alpha$ und daraus unmittelbar $1-cd=0$. Da $O_K$ auch noch Integritätsring und somit nullteilerfrei ist, folgt $\alpha=0$  ($\beta =0$) oder $1-cd=0$. Dies spricht dafür, dass $b,c \in O_K^{\times}$ also Einheiten sind. So sind $\alpha$ und $\beta$ assoziiert.
\\
So ist obige Abbildung wohldefiniert und injektiv.\\
\\
Beweis der Surjektivität:
Seien dazu $\beta \in \mathfrak{L}\backslash{0}$ und  $\langle 0 \rangle \subsetneq \mathfrak{J} \subseteq O_K$ mit $\mathfrak{J} \in C$ beliebig und weiter $\mathfrak{J\cdot L}= \langle \alpha \rangle$ für ein $\alpha \in \mathfrak{L}$. Dann gilt $\frac{\beta}{\alpha} \cdot \mathfrak{J} \in C$ und $\frac{\beta}{\alpha} \cdot \mathfrak{J\cdot L} = \langle \beta\rangle$. Diese Darstellung ist richtig, da für $\frac{\beta}{\alpha} \cdot \mathfrak{J} \subseteq K$ tatsächlich gilt $\frac{\beta}{\alpha} \cdot \mathfrak{J} \subseteq O_K$. Dies folgt daraus, dass $\beta \in \mathfrak{L}$ und damit $\langle \beta \rangle \subseteq \mathfrak{L}$. Das wiederum hat zur Folge, dass $\mathfrak{L}\mid \langle \beta \rangle$ ist. Dies bedeutet, dass ein Ideal $\mathfrak{L'}\subseteq O_K$ mit $\mathfrak{L\cdot L'}= \langle \beta \rangle$. Durch Einsetzen in $\frac{\beta}{\alpha} \cdot \mathfrak{J\cdot L} = \langle \beta\rangle$ ist $\frac{\beta}{\alpha} \cdot \mathfrak{J\cdot L} = \mathfrak{L \cdot L'} $ und nach Multiplikation mit $\mathfrak{L^{-1}}$ folgt $\frac{\beta}{\alpha} \cdot \mathfrak{L'}\subseteq O_K$. Daraus folgt die Surjektivität.\\
\\
Die Eigenschaften der Norm $$N(\alpha)= N_{K\mid \Q}(\langle \alpha \rangle)=N_{K\mid \Q}(\mathfrak{J\cdot L})=N_{K\mid \Q}(\mathfrak{J})\cdot N_{K\mid \Q}(\mathfrak{L})$$ ermöglichen, die Bijektion umzuschreiben. Aus Vortrag 5 ist bekannt, dass stets $\gamma, \delta \in \mathfrak{L}$ mit $$\mathfrak{L}= \Z\gamma+\Z\delta = \{x\gamma+y\delta\colon x,y \in \Z\}$$ gefunden werden können. Weiter ist aus Vortrag 7 gegeben, dass $$f:=\frac{N(\gamma X+\delta Y)}{N_{K\mid \Q}(\mathfrak{L})}=\frac{(\gamma X + \delta Y)\cdot \iota((\gamma)X+\iota(\delta)Y)}{N_{K\mid \Q}(\mathfrak{L})} \in \Z[X,Y]$$eine primitive, positiv definite quadratische Form mit Diskriminante $\Delta_K$ ist, die unter der Bijektion $C(\Delta_K)\rightarrow Cl_K$ auf $C^{-1}$ (die Klasse von $\mathfrak{L}$ in $Cl_K$) abgebildet wird. Hieraus folgt, dass $$\sum_{\mathfrak{J}\in C}\frac{1}{N_{K\mid \Q }(\mathfrak{J})^s}=\frac{1}{|O_K^{\times}|}\sum_{(x,y)\in \Z^2\backslash \{ (0,0)\}}\frac{1}{f(x,y)^s}$$wobei sich dies durch das Ersetzen von $f$ durch eine reduzierte quadratische Form, die eigentlich äquivalent zu $f$ ist, nicht ändert. Das folgt aus der Bemerkung nach Cox \cite{Cox}[S. 23], dass äquivalente Formen dieselben ganzen Zahlen repräsentieren.
\\
Seien also $f_1,...,f_k \in \Z[X,Y]$ reduzierte quadratische Formen, die die Äquivalenzklassen aus $C(\Delta_K)$ repräsentieren. Dann erhalten wir aufgrund absoluter Konvergenz, dass $$\zeta_K(s) = \sum_{j=1}^{k}\frac{1}{|O_K^{\times}|}\sum_{(x,y) \in \Z^2 \backslash \{(0,0)\}}\frac{1}{f_j(x,y)^s}=\frac{1}{|O_K^{\times}|}\sum_{n=1}^{\infty}\frac{r_K(n)}{n^s}$$
Das Gleichheitszeichen folgt dann analog mit dem vorherigen Lemma.
\end{proof}
\end{lem}
 
\begin{thebibliography}{99}
\bibitem{Neukirch} J. Neukirch: \textit{Algebraische Zahlentheorie}. Springer, 1. Auflage (1992)
\bibitem{Modler} F. Modler, M. Kreh: \textit{Tutorium Algebra}. Springer, 3. Auflage (2013)
\bibitem{Marcus} D. A. Marcus: \textit{Number Fields}. Springer, 2. Auflage (2018)
\bibitem{Freitag} E. Freitag, R. Busam: \textit{Funktionentheorie 1}. Springer, 4. Auflage (2006)
\bibitem{Cox} D. A. Cox: \textit{Primes of the Form ${x}^2 + n{y}^2$}. Wiley, 2. Auflage (2013)
\bibitem{Baxa} C. Baxa: \textit{Vorlesungsskript Zahlentheorie}. Universität Wien
\end{thebibliography}

\end{document}